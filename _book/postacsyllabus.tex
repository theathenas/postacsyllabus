\documentclass[]{book}
\usepackage{lmodern}
\usepackage{amssymb,amsmath}
\usepackage{ifxetex,ifluatex}
\usepackage{fixltx2e} % provides \textsubscript
\ifnum 0\ifxetex 1\fi\ifluatex 1\fi=0 % if pdftex
  \usepackage[T1]{fontenc}
  \usepackage[utf8]{inputenc}
\else % if luatex or xelatex
  \ifxetex
    \usepackage{mathspec}
  \else
    \usepackage{fontspec}
  \fi
  \defaultfontfeatures{Ligatures=TeX,Scale=MatchLowercase}
\fi
% use upquote if available, for straight quotes in verbatim environments
\IfFileExists{upquote.sty}{\usepackage{upquote}}{}
% use microtype if available
\IfFileExists{microtype.sty}{%
\usepackage{microtype}
\UseMicrotypeSet[protrusion]{basicmath} % disable protrusion for tt fonts
}{}
\usepackage[margin=1in]{geometry}
\usepackage{hyperref}
\hypersetup{unicode=true,
            pdftitle={Post Academic Syllabus},
            pdfauthor={Beth M. Duckles},
            pdfborder={0 0 0},
            breaklinks=true}
\urlstyle{same}  % don't use monospace font for urls
\usepackage{natbib}
\bibliographystyle{apalike}
\usepackage{color}
\usepackage{fancyvrb}
\newcommand{\VerbBar}{|}
\newcommand{\VERB}{\Verb[commandchars=\\\{\}]}
\DefineVerbatimEnvironment{Highlighting}{Verbatim}{commandchars=\\\{\}}
% Add ',fontsize=\small' for more characters per line
\usepackage{framed}
\definecolor{shadecolor}{RGB}{248,248,248}
\newenvironment{Shaded}{\begin{snugshade}}{\end{snugshade}}
\newcommand{\KeywordTok}[1]{\textcolor[rgb]{0.13,0.29,0.53}{\textbf{#1}}}
\newcommand{\DataTypeTok}[1]{\textcolor[rgb]{0.13,0.29,0.53}{#1}}
\newcommand{\DecValTok}[1]{\textcolor[rgb]{0.00,0.00,0.81}{#1}}
\newcommand{\BaseNTok}[1]{\textcolor[rgb]{0.00,0.00,0.81}{#1}}
\newcommand{\FloatTok}[1]{\textcolor[rgb]{0.00,0.00,0.81}{#1}}
\newcommand{\ConstantTok}[1]{\textcolor[rgb]{0.00,0.00,0.00}{#1}}
\newcommand{\CharTok}[1]{\textcolor[rgb]{0.31,0.60,0.02}{#1}}
\newcommand{\SpecialCharTok}[1]{\textcolor[rgb]{0.00,0.00,0.00}{#1}}
\newcommand{\StringTok}[1]{\textcolor[rgb]{0.31,0.60,0.02}{#1}}
\newcommand{\VerbatimStringTok}[1]{\textcolor[rgb]{0.31,0.60,0.02}{#1}}
\newcommand{\SpecialStringTok}[1]{\textcolor[rgb]{0.31,0.60,0.02}{#1}}
\newcommand{\ImportTok}[1]{#1}
\newcommand{\CommentTok}[1]{\textcolor[rgb]{0.56,0.35,0.01}{\textit{#1}}}
\newcommand{\DocumentationTok}[1]{\textcolor[rgb]{0.56,0.35,0.01}{\textbf{\textit{#1}}}}
\newcommand{\AnnotationTok}[1]{\textcolor[rgb]{0.56,0.35,0.01}{\textbf{\textit{#1}}}}
\newcommand{\CommentVarTok}[1]{\textcolor[rgb]{0.56,0.35,0.01}{\textbf{\textit{#1}}}}
\newcommand{\OtherTok}[1]{\textcolor[rgb]{0.56,0.35,0.01}{#1}}
\newcommand{\FunctionTok}[1]{\textcolor[rgb]{0.00,0.00,0.00}{#1}}
\newcommand{\VariableTok}[1]{\textcolor[rgb]{0.00,0.00,0.00}{#1}}
\newcommand{\ControlFlowTok}[1]{\textcolor[rgb]{0.13,0.29,0.53}{\textbf{#1}}}
\newcommand{\OperatorTok}[1]{\textcolor[rgb]{0.81,0.36,0.00}{\textbf{#1}}}
\newcommand{\BuiltInTok}[1]{#1}
\newcommand{\ExtensionTok}[1]{#1}
\newcommand{\PreprocessorTok}[1]{\textcolor[rgb]{0.56,0.35,0.01}{\textit{#1}}}
\newcommand{\AttributeTok}[1]{\textcolor[rgb]{0.77,0.63,0.00}{#1}}
\newcommand{\RegionMarkerTok}[1]{#1}
\newcommand{\InformationTok}[1]{\textcolor[rgb]{0.56,0.35,0.01}{\textbf{\textit{#1}}}}
\newcommand{\WarningTok}[1]{\textcolor[rgb]{0.56,0.35,0.01}{\textbf{\textit{#1}}}}
\newcommand{\AlertTok}[1]{\textcolor[rgb]{0.94,0.16,0.16}{#1}}
\newcommand{\ErrorTok}[1]{\textcolor[rgb]{0.64,0.00,0.00}{\textbf{#1}}}
\newcommand{\NormalTok}[1]{#1}
\usepackage{longtable,booktabs}
\usepackage{graphicx,grffile}
\makeatletter
\def\maxwidth{\ifdim\Gin@nat@width>\linewidth\linewidth\else\Gin@nat@width\fi}
\def\maxheight{\ifdim\Gin@nat@height>\textheight\textheight\else\Gin@nat@height\fi}
\makeatother
% Scale images if necessary, so that they will not overflow the page
% margins by default, and it is still possible to overwrite the defaults
% using explicit options in \includegraphics[width, height, ...]{}
\setkeys{Gin}{width=\maxwidth,height=\maxheight,keepaspectratio}
\IfFileExists{parskip.sty}{%
\usepackage{parskip}
}{% else
\setlength{\parindent}{0pt}
\setlength{\parskip}{6pt plus 2pt minus 1pt}
}
\setlength{\emergencystretch}{3em}  % prevent overfull lines
\providecommand{\tightlist}{%
  \setlength{\itemsep}{0pt}\setlength{\parskip}{0pt}}
\setcounter{secnumdepth}{5}
% Redefines (sub)paragraphs to behave more like sections
\ifx\paragraph\undefined\else
\let\oldparagraph\paragraph
\renewcommand{\paragraph}[1]{\oldparagraph{#1}\mbox{}}
\fi
\ifx\subparagraph\undefined\else
\let\oldsubparagraph\subparagraph
\renewcommand{\subparagraph}[1]{\oldsubparagraph{#1}\mbox{}}
\fi

%%% Use protect on footnotes to avoid problems with footnotes in titles
\let\rmarkdownfootnote\footnote%
\def\footnote{\protect\rmarkdownfootnote}

%%% Change title format to be more compact
\usepackage{titling}

% Create subtitle command for use in maketitle
\newcommand{\subtitle}[1]{
  \posttitle{
    \begin{center}\large#1\end{center}
    }
}

\setlength{\droptitle}{-2em}
  \title{Post Academic Syllabus}
  \pretitle{\vspace{\droptitle}\centering\huge}
  \posttitle{\par}
  \author{Beth M. Duckles}
  \preauthor{\centering\large\emph}
  \postauthor{\par}
  \predate{\centering\large\emph}
  \postdate{\par}
  \date{2018-06-03}

\usepackage{booktabs}

\begin{document}
\maketitle

{
\setcounter{tocdepth}{1}
\tableofcontents
}
\begin{Shaded}
\begin{Highlighting}[]
\KeywordTok{options}\NormalTok{(}\DataTypeTok{tinytex.verbose =} \OtherTok{TRUE}\NormalTok{)}
\end{Highlighting}
\end{Shaded}

\chapter{Introduction}\label{intro}

There is likely a very good reason you've decided to search the internet
for advice on leaving academia, been sent a link to this or have
stumbled on this syllabus. Regardless of your stage of being in the
academy, the way you've gotten here or your feelings about wanting to
leave, you are welcome here. We probably don't have all the answers for
you, but we have a few thoughts. And we certainly know what it's like to
be where you are.

There are a lot of external reasons that could have come your way that
would make this website needed. Maybe the job search process decided for
you and you're not going to get that job you wanted. Maybe you're a grad
student who knows she needs options. Maybe you're a tenured professor
who doesn't want to stay. Maybe you're not sure but just want to know.
Whatever stage you're at, you're welcome here.

You'll also likely have a lot of emotions about this too. We'll talk
about that in Ch ?. Just know that those of us who have left we get it.

\section{A Syllabus? Really?}\label{a-syllabus-really}

Well, why not? We use syllabi to teach things to our students, why
wouldn't we put together a framework for learning more about the job of
leaving the academy? There are a few reasons why I chose this as a
format. 1) Because you know what a syllabus is about if you're an
academic. You have taken classes and read other people's syllabi. You
might have written your own. This is comfortable. 2) When making your
own classes you probably look at other people's syllabi and ``borrow''
from others. This is not only acceptable, it's normal. Same here. Take
what you want, leave the rest. 3) Leaving academia and getting together
the skills to make this leap is no different than any other class you've
taken. It's just work. A syllabus reminds you that this is doable. 4) A
syllabus is a living document and it changes from semester to semester.
I'm hopeful that this will do the same. I also strongly encourage you to
add to it, help me edit it as we go. I need what you know too.

\section{Prerequisite}\label{prerequisite}

This syllabus expects that you have spent time in a graduate program
that is designed to prepare you for a career in academia. This course
also presumes that you have asked or are currently open to the question
of what life would be like to not be an academic.

\section{About}\label{about}

This syllabus is designed by me Beth Duckles, but it would not exist
without the incredible group of women who have joined the Athenas Slack
Channel for Post Ac women and the people who responded to a survey I put
out for resources on Post Ac life.

This is a living document and while I take responsibility for the
opinions inside, I strongly encourage additional resources, materials,
ideas for assignments, or any other feedback you have. I know there's
more, please share any resources that you think might fit with me here.

\chapter{Quit Lit}\label{quit-lit}

Quit lit is rarely defined but most academics have seen or heard of its
prominence in recent years. These documents are a particular type of
discussion of the challenges in higher education. They often take a
first person perspective and reflect both on the individual's story and
how this connects to the larger questions within the academy and higher
education. By starting with an incomplete typology of perspectives on
why people leave the academy, we get a sense of the variety of stories
that are told and can reflect on how these themes may emerge in our own
stories.

The point is to begin to own one's story and to take seriously how our
own individual positions and experiences are connected to the larger
world. Being willing to examine our own story in light of the larger
structure, industry and mechanisms that we have been a part of can ease
the individual burden and begin the healing process.

\section{Readings}\label{readings}

\begin{itemize}
\tightlist
\item
  \href{https://www.theatlantic.com/entertainment/archive/2015/09/dont-quit-your-day-job/404671/}{Garber,
  Megan. ``The Rise of Quit Lit'' - Atlantic}
\item
  \href{http://www.slate.com/articles/life/culturebox/2013/04/there_are_no_academic_jobs_and_getting_a_ph_d_will_make_you_into_a_horrible.html}{Schuman,
  Rebecca ``Thesis Hatement'' - Slate}
\item
  \href{http://erinbartram.com/uncategorized/the-sublimated-grief-of-the-left-behind/}{Bartram,
  Erin - ``The Sublimated Grief of the Left Behind''}
\item
  \href{http://www.alicolleenneff.com/blog/2017/11/8/on-academic-precarity}{Neff,
  Ali Colleen - ``On Academic Precarity''}
\item
  \href{\%20https://www.vox.com/2015/9/8/9261531/professor-quitting-job}{Lee,
  Oliver - ``I have one of the best jobs in Academia. Here's why I'm
  walking away'' - Vox}
\item
  \href{\%20https://conditionallyaccepted.com/2015/06/16/quit/}{Conditionally
  Accepted, ``Dear Department, I Quit.''}
\item
  \href{\%20https://www.allisonharbin.com/post-phd/why-i-left-academia-part-1}{Harbin,
  Alison - ``Why I Left Academia Part I \& II''}
\item
  \href{https://chroniclevitae.com/news/216-why-so-many-academics-quit-and-tell}{Dunn,
  Sydni - ``Why So Many Academics Quit and Tell''}
\item
  For more, see the list of pieces in a Google
  Doc:\href{\%20https://docs.google.com/spreadsheets/d/1OODoiZKeAtiGiI3IAONCspryCHWo5Yw9xkQzkRntuMU/edit\#gid=0}{''Quit
  Lit: The Vitae List''}
\end{itemize}

\section{Homework:}\label{homework}

\begin{enumerate}
\def\labelenumi{\arabic{enumi})}
\tightlist
\item
  After reading the above quit lit, find more that fits with your
  experiences. Consider the themes and note which ones fit best with
  your story.
\end{enumerate}

Chose among any combination of the following topics/themes: - academic
precarity and economic instability - the contingent labor market and
low-paid adjunct positions - not enough tenure track jobs and the
competition for tenure track positions, academia is not for me, academia
is for me but I didn't get a job, academia is for me but I hate
teaching, unfair teaching burdens, student apathy, student entitlement,
unprepared students, student as customer, the rise of online education,
increased student tuition, student loan bloat, the broken hiring
process, the broken tenure process, the broken academic publishing
process, decreased university funding, increased higher education
administration, low faculty salaries, declining funding for research,
the declining liberal arts, anti-intellectualism, classism, geographic
isolation, loneliness, grief, the ``two body'' problem, having a child,
having a child with special needs, having small children, incompatible
careers, family illness, divorce, everyday institutional racism and
microaggressions, mental illness, workaholism, everyday sexism, sexual
harrassment, sexual assault, assault, verbal abuse, gaslighting, unequal
emotional labor workload, illegal behavior, institutional politics and
infighting, sabotage, the trap of post doctoral positions, the trap of
visiting assistant positions, the trap of adjuncting, dropping out of a
Ph.D program, the desire to do work that matters, the desire to do
manual work or the desire to have a life.

\begin{enumerate}
\def\labelenumi{\arabic{enumi})}
\setcounter{enumi}{1}
\tightlist
\item
  Write your own quit lit piece. If you have not already left, imagine
  you have or will soon leave. If you have left, make your writing
  cathartic.
\end{enumerate}

Optional Extra credit: Let someone (or a lot of someones) read your quit
lit piece.

\chapter{You Are Not Alone}\label{you-are-not-alone}

This class will remind you that there are others who have been
successful in doing this and actually, there is some evidence that you
will be happier if you do leave the academy.

\section{Resources}\label{resources}

\begin{itemize}
\tightlist
\item
  Lorem ipsum dolor sit amet, consectetur adipiscing elit, sed do
  eiusmod tempor incididunt ut labore et dolore magna aliqua. Ut enim ad
  minim veniam, quis nostrud exercitation ullamco laboris nisi ut
  aliquip ex ea commodo consequat. Duis aute irure dolor in
  reprehenderit in voluptate velit esse cillum dolore eu fugiat nulla
  pariatur. Excepteur sint occaecat cupidatat non proident, sunt in
  culpa qui officia deserunt mollit anim id est laborum.
\end{itemize}

\section{Readings}\label{readings-1}

\begin{itemize}
\tightlist
\item
  What I wish I had Known - Beth M. Duckles
\item
  \href{http://theprofessorisin.com/}{The Professor is In}
\item
  \href{https://www.imaginephd.com/}{Imagine Ph.D.}
\item
  \href{https://community.beyondprof.com/}{Beyond the Professoriate}
\item
  \href{https://www.timeshighereducation.com/blog/what-happens-when-academics-quit-good-things-it-turns-out}{Perel,
  Greta - ``What Happens When Academics Quit? Good Things it Turns
  Out.''}
\end{itemize}

\chapter{Networking}\label{networking}

This class will demystify and simplify the task of networking by
encouraging you to note your strengths (yes introverts have networking
strengths) and to give you a chance to play around with and consider how
best to network into new ideas.

The truth is, that getting a new job, starting a new career, or even
being a public academic requires networking. I am not suggesting that
you change who you are, or do anything that turns you into a smarmy
salesperson. You gotta be you.

\begin{itemize}
\tightlist
\item
  \href{https://code.likeagirl.io/networking-strategies-for-beginners-986fe0b3efdf}{``Bajuniemi,
  Abby - ``Networking Strategies for Beginners''}
\item
  Linkedin
\item
  Meetup.com
\item
  Twitter
\item
  Facebook Groups
\end{itemize}

\section{Homework}\label{homework-1}

\begin{enumerate}
\def\labelenumi{\arabic{enumi})}
\tightlist
\item
  Create social media profiles for Linked In, Twitter and Meetup. If you
  dislike or are nervous about being on one or more of these platforms
  consider creating very specific boundaries around how you choose to
  use them. For instance, you might decide that you will not friend
  people on Linked in that were in your classes. Or you might only tweet
  things that you think people who are in your field would be interested
  in. Or you might decide
\end{enumerate}

2a) Connect with five new people on LinkedIn that you already know.
Write them a short personal note saying something along the lines of:
``It's great to see you on here, hope you are doing well. - YourName''
or ``I was thinking about you the other day and it's good to look you up
on here and see you're doing well.'' - YourName or ``I saw this
article/anecdote/event/meme and thought of you. Hope you're doing
well.'' - YourName

2b) Follow someone you think is interesting on Twitter. Read their
tweets.

2c) Look for some Meetups you might consider going to. Put one on the
calendar.

2d) Join a new Facebook Group and respond to or ``like'' someone's post.

\chapter{A Job}\label{a-job}

There are a thousand and one books out there on how to find a job and a
ton of people who can help. That said, it can be somewhat overwhelming
to approach those when you're an academic. We face very specific
challenges like - how do I turn my CV into a resume? or how do I talk to
people about my research without boring them to tears? or what on earth
do I have to share that's marketable? I have a few brief thoughts about
the act of trying to find a job after being an academic, but there are
resources.

First, recognize that you have skills. Even if you have the most
esoteric dissertation in the world, you have skills and often can teach
them or use them in service to all kinds of projects. Consider what you
did love about academia. Was it reading? Writing? Teaching? Was it the
content you were working on? Was it the methods you were using? Did you
like the meetings or prefer to hide out in the lab/library/archives?
Consider your answers to these questions as you look for your job and
try to find places where those skills are valued.

Second, the path of finding your post ac job can be trial and error. It
often does not follow a clear path and there isn't anyone else who can
know what the next step is for you to take. It can be bewildering,
frustrating and incredibly scary for academics who have had the path
laid out for them (take these classes, then do these exams, then write
this paper etc). This kind of figure-it-out-as-you-go mentality is
actually normal for non academics. If you need to be reminded of this,
go talk to a friend who never went to grad school and ask them how they
got their current job. You might be surprised.

Third, you gotta pay your bills. Don't discard jobs that are completely
different than your academic work, if they fit for you and make you
happy. There is no shame in doing work that is non academic to make ends
meet. Do side gigs if you need to, start up an etsy shop, work with high
school kids on their SAT prep, teach art classes, do freelance writing
or editing. Consider arranging flowers, being a cheesemonger, becoming a
sommelier, a bike mechanic, doing community organizing, running online
seminars, Let me reiterate, there is no shame in any of these jobs. Or
rather, you can feel ashamed if you really want to but you might
consider that most people are not judging you.

\begin{itemize}
\tightlist
\item
  \href{http://jobsontoast.com/}{Jobs on Toast}
\item
  \href{https://versatilephd.com/}{Versatile Ph.D.}
\item
  \url{https://medium.com/design-leadership/benefits-of-a-non-linear-design-career-path-d0add49a90c8}
\item
  Side gig book - Chris Guillibeau
\item
  \url{https://www.insidehighered.com/news/2017/12/18/study-humanities-and-social-science-phds-working-outside-academe-are-happier-their?utm_source=Inside+Higher+Ed}
\end{itemize}

\section{Homework}\label{homework-2}

\begin{itemize}
\tightlist
\item
  Write down ten jobs you would like to do. Don't think about it, just
  write them down. Doesn't matter if they're silly or strange. Feel free
  to put circus performer, professional basketweaver or person who does
  nature walks for a living. Do this every day for a week.
\item
  Ask someone you trust to tell you what they think you're good at.
  Listen without interrupting. Write down what they say.
\end{itemize}

\chapter{Creativity}\label{creativity}

It may seem strange for me to have a week on creativity but in my own
recovery from academia I've found it to be an incredibly potent tool for
dealing with what I feel and think as well as a way for me to feel more
content and happy in my life.

I believe we are all creative. I don't think there are any among us who
are not creative. For many academics, the creative urges have been
quashed by the desire to please those who we have looked up to in our
school years. We may also believe ourselves not to be creative, or to
have taken to hear the kind of language that tells us that a ``true''
scientist/historian/professor isn't a creative person.

My number one recommendation is Julia Cameron's ``The Artist's Way''. In
that book she describes much of what has caused many of us academics a
lot of pain. She does so with compassion and grace and she slowly,
steadily and lovingly encourages the nascent artists inside of us out of
their shell.

\begin{itemize}
\tightlist
\item
  Julia Cameron ``The Artist's Way''
\item
  SARK's book
\end{itemize}

\section{Homework}\label{homework-3}

\begin{itemize}
\tightlist
\item
  Do one thing that seems fun and creative. A few ideas: Play with
  legos, cook a meal, doodle on a piece of scrap paper until the page is
  filled up, write bad poetry about something you can see in the room
  you're in now, take a train, take a pottery class, bus or car trip and
  spend the entire trip looking out of the window, hum a song, play an
  instrument, drum on the tabletop, paint, garden, sew, knit, crochet,
  find a youtube video that teaches you a dance step, make snow/sand
  angels, make splashes in the water and watch the shape they make, eat
  your favorite childhood snack, walk somewhere interesting, skip
  stones, build a little teepee out of sticks in the woods, make a fire,
  do karaoke, read a poem or a speech out loud, watch a movie, make
  paper airplanes, play with arduino, go see a play, do a zumba class,
  learn a few words in another language, watch a dance performance and
  anything else your heart desires.
\end{itemize}

\chapter{Entrepreneurship for Post
Academics}\label{entrepreneurship-for-post-academics}

Freelancers Union
\href{https://www.freelancersunion.org/}{~{[}{]}https://www.freelancersunion.org/}
- Being Boss Podcast - Your local small business development council
(SBDC) - SCORE office \url{https://www.score.org/} - \$100 Startup
Marketing - Seth Godin - Books/Marketing Seminar/Akimbo Podcast -
Marketing Mentor Podcast - We Write You - How to Write Like a Person\\
- Leonie Dawson's Workbook - Biz
\url{https://shiningacademy.com/2018-life-and-business-goals-workbooks-and-diary-planners-by-leonie-dawson/}
Tools - \url{https://app.and.co/}

\chapter{Industry Specific Resources}\label{industry-specific-resources}

\section{Writing/Publishing}\label{writingpublishing}

\begin{itemize}
\tightlist
\item
  www.loft.org - writing classes
\item
  www.pw.org
\item
  \url{https://www.authorspublish.com/}
\item
  \url{http://www.theinternationalfreelancer.com/}
\end{itemize}

\section{Linguistics}\label{linguistics}

\begin{itemize}
\tightlist
\item
  careerlinguist.com
\end{itemize}

\section{Social Scientists}\label{social-scientists}

\begin{itemize}
\tightlist
\item
  \url{https://www.epicpeople.org/}
\end{itemize}

\section{Other Resources}\label{other-resources}

\begin{itemize}
\tightlist
\item
  herc.com
\item
  The Artful Adjunct
\item
  The Art of Nonconformity
\end{itemize}

\chapter{Money}\label{money}

\begin{itemize}
\tightlist
\item
  YNAB - You Need a Budget
\item
  Overcoming Underearning, Barbara Stanny
\item
  Chelsea Fagan - Financial Diet
\item
  JD Roth - ``Your Money The Missing Manual''
\item
  Your money or your life.
\item
  Personal Finance for PhDs \url{http://pfforphds.com/}
\item
  Negotiation -
  \url{http://www.singlefounder.com/tips-on-negotiating-a-great-consulting-rate/}
\end{itemize}

\section{Articles/Stories}\label{articlesstories}

\begin{itemize}
\tightlist
\item
  \url{http://erinbartram.com/uncategorized/the-sublimated-grief-of-the-left-behind/}
\end{itemize}

\bibliography{book.bib,packages.bib}


\end{document}
