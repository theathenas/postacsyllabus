\documentclass[]{book}
\usepackage{lmodern}
\usepackage{amssymb,amsmath}
\usepackage{ifxetex,ifluatex}
\usepackage{fixltx2e} % provides \textsubscript
\ifnum 0\ifxetex 1\fi\ifluatex 1\fi=0 % if pdftex
  \usepackage[T1]{fontenc}
  \usepackage[utf8]{inputenc}
\else % if luatex or xelatex
  \ifxetex
    \usepackage{mathspec}
  \else
    \usepackage{fontspec}
  \fi
  \defaultfontfeatures{Ligatures=TeX,Scale=MatchLowercase}
\fi
% use upquote if available, for straight quotes in verbatim environments
\IfFileExists{upquote.sty}{\usepackage{upquote}}{}
% use microtype if available
\IfFileExists{microtype.sty}{%
\usepackage{microtype}
\UseMicrotypeSet[protrusion]{basicmath} % disable protrusion for tt fonts
}{}
\usepackage[margin=1in]{geometry}
\usepackage{hyperref}
\hypersetup{unicode=true,
            pdftitle={Post Academic Syllabus},
            pdfauthor={Beth M. Duckles},
            pdfborder={0 0 0},
            breaklinks=true}
\urlstyle{same}  % don't use monospace font for urls
\usepackage{natbib}
\bibliographystyle{apalike}
\usepackage{color}
\usepackage{fancyvrb}
\newcommand{\VerbBar}{|}
\newcommand{\VERB}{\Verb[commandchars=\\\{\}]}
\DefineVerbatimEnvironment{Highlighting}{Verbatim}{commandchars=\\\{\}}
% Add ',fontsize=\small' for more characters per line
\usepackage{framed}
\definecolor{shadecolor}{RGB}{248,248,248}
\newenvironment{Shaded}{\begin{snugshade}}{\end{snugshade}}
\newcommand{\KeywordTok}[1]{\textcolor[rgb]{0.13,0.29,0.53}{\textbf{#1}}}
\newcommand{\DataTypeTok}[1]{\textcolor[rgb]{0.13,0.29,0.53}{#1}}
\newcommand{\DecValTok}[1]{\textcolor[rgb]{0.00,0.00,0.81}{#1}}
\newcommand{\BaseNTok}[1]{\textcolor[rgb]{0.00,0.00,0.81}{#1}}
\newcommand{\FloatTok}[1]{\textcolor[rgb]{0.00,0.00,0.81}{#1}}
\newcommand{\ConstantTok}[1]{\textcolor[rgb]{0.00,0.00,0.00}{#1}}
\newcommand{\CharTok}[1]{\textcolor[rgb]{0.31,0.60,0.02}{#1}}
\newcommand{\SpecialCharTok}[1]{\textcolor[rgb]{0.00,0.00,0.00}{#1}}
\newcommand{\StringTok}[1]{\textcolor[rgb]{0.31,0.60,0.02}{#1}}
\newcommand{\VerbatimStringTok}[1]{\textcolor[rgb]{0.31,0.60,0.02}{#1}}
\newcommand{\SpecialStringTok}[1]{\textcolor[rgb]{0.31,0.60,0.02}{#1}}
\newcommand{\ImportTok}[1]{#1}
\newcommand{\CommentTok}[1]{\textcolor[rgb]{0.56,0.35,0.01}{\textit{#1}}}
\newcommand{\DocumentationTok}[1]{\textcolor[rgb]{0.56,0.35,0.01}{\textbf{\textit{#1}}}}
\newcommand{\AnnotationTok}[1]{\textcolor[rgb]{0.56,0.35,0.01}{\textbf{\textit{#1}}}}
\newcommand{\CommentVarTok}[1]{\textcolor[rgb]{0.56,0.35,0.01}{\textbf{\textit{#1}}}}
\newcommand{\OtherTok}[1]{\textcolor[rgb]{0.56,0.35,0.01}{#1}}
\newcommand{\FunctionTok}[1]{\textcolor[rgb]{0.00,0.00,0.00}{#1}}
\newcommand{\VariableTok}[1]{\textcolor[rgb]{0.00,0.00,0.00}{#1}}
\newcommand{\ControlFlowTok}[1]{\textcolor[rgb]{0.13,0.29,0.53}{\textbf{#1}}}
\newcommand{\OperatorTok}[1]{\textcolor[rgb]{0.81,0.36,0.00}{\textbf{#1}}}
\newcommand{\BuiltInTok}[1]{#1}
\newcommand{\ExtensionTok}[1]{#1}
\newcommand{\PreprocessorTok}[1]{\textcolor[rgb]{0.56,0.35,0.01}{\textit{#1}}}
\newcommand{\AttributeTok}[1]{\textcolor[rgb]{0.77,0.63,0.00}{#1}}
\newcommand{\RegionMarkerTok}[1]{#1}
\newcommand{\InformationTok}[1]{\textcolor[rgb]{0.56,0.35,0.01}{\textbf{\textit{#1}}}}
\newcommand{\WarningTok}[1]{\textcolor[rgb]{0.56,0.35,0.01}{\textbf{\textit{#1}}}}
\newcommand{\AlertTok}[1]{\textcolor[rgb]{0.94,0.16,0.16}{#1}}
\newcommand{\ErrorTok}[1]{\textcolor[rgb]{0.64,0.00,0.00}{\textbf{#1}}}
\newcommand{\NormalTok}[1]{#1}
\usepackage{longtable,booktabs}
\usepackage{graphicx,grffile}
\makeatletter
\def\maxwidth{\ifdim\Gin@nat@width>\linewidth\linewidth\else\Gin@nat@width\fi}
\def\maxheight{\ifdim\Gin@nat@height>\textheight\textheight\else\Gin@nat@height\fi}
\makeatother
% Scale images if necessary, so that they will not overflow the page
% margins by default, and it is still possible to overwrite the defaults
% using explicit options in \includegraphics[width, height, ...]{}
\setkeys{Gin}{width=\maxwidth,height=\maxheight,keepaspectratio}
\IfFileExists{parskip.sty}{%
\usepackage{parskip}
}{% else
\setlength{\parindent}{0pt}
\setlength{\parskip}{6pt plus 2pt minus 1pt}
}
\setlength{\emergencystretch}{3em}  % prevent overfull lines
\providecommand{\tightlist}{%
  \setlength{\itemsep}{0pt}\setlength{\parskip}{0pt}}
\setcounter{secnumdepth}{5}
% Redefines (sub)paragraphs to behave more like sections
\ifx\paragraph\undefined\else
\let\oldparagraph\paragraph
\renewcommand{\paragraph}[1]{\oldparagraph{#1}\mbox{}}
\fi
\ifx\subparagraph\undefined\else
\let\oldsubparagraph\subparagraph
\renewcommand{\subparagraph}[1]{\oldsubparagraph{#1}\mbox{}}
\fi

%%% Use protect on footnotes to avoid problems with footnotes in titles
\let\rmarkdownfootnote\footnote%
\def\footnote{\protect\rmarkdownfootnote}

%%% Change title format to be more compact
\usepackage{titling}

% Create subtitle command for use in maketitle
\newcommand{\subtitle}[1]{
  \posttitle{
    \begin{center}\large#1\end{center}
    }
}

\setlength{\droptitle}{-2em}

  \title{Post Academic Syllabus}
    \pretitle{\vspace{\droptitle}\centering\huge}
  \posttitle{\par}
    \author{Beth M. Duckles}
    \preauthor{\centering\large\emph}
  \postauthor{\par}
      \predate{\centering\large\emph}
  \postdate{\par}
    \date{2018-08-02}

\usepackage{booktabs}

\usepackage{amsthm}
\newtheorem{theorem}{Theorem}[chapter]
\newtheorem{lemma}{Lemma}[chapter]
\theoremstyle{definition}
\newtheorem{definition}{Definition}[chapter]
\newtheorem{corollary}{Corollary}[chapter]
\newtheorem{proposition}{Proposition}[chapter]
\theoremstyle{definition}
\newtheorem{example}{Example}[chapter]
\theoremstyle{definition}
\newtheorem{exercise}{Exercise}[chapter]
\theoremstyle{remark}
\newtheorem*{remark}{Remark}
\newtheorem*{solution}{Solution}
\begin{document}
\maketitle

{
\setcounter{tocdepth}{1}
\tableofcontents
}
\begin{Shaded}
\begin{Highlighting}[]
\KeywordTok{options}\NormalTok{(}\DataTypeTok{tinytex.verbose =} \OtherTok{TRUE}\NormalTok{)}
\end{Highlighting}
\end{Shaded}

\chapter{Introduction}\label{introduction}

There is likely a very good reason you've decided to search the internet
for advice on leaving academia, been sent a link to this or have
stumbled on this syllabus. Regardless of your stage of being in the
academy, the way you've gotten here or your feelings about wanting to
leave, you are welcome here. We probably don't have all the answers for
you, but we have a few thoughts. And we certainly know what it's like to
be where you are.

Why are you here? Well there are a lot of reasons you might have decided
to take a look.\\
- Maybe you're at the tail end of a job search and you're not going to
get a job you want. - Maybe you're a grad student who knows she needs
options. - Maybe you're a tenured professor who doesn't want to stay in
academia for any number of reasons.\\
- Maybe you're not sure but just want to know what your options are.

Whatever stage you're at, you're welcome here.

You'll also likely have a lot of emotions about this too. We'll talk
about that in Ch ?. Just know that we get it.

\section{A Syllabus? Really?}\label{a-syllabus-really}

Well, why not? We use syllabi to teach content to students, why wouldn't
we put together a framework for learning more about the job of leaving
the academy?

There are a few reasons why I chose this as a format.

\begin{enumerate}
\def\labelenumi{\arabic{enumi})}
\item
  Because you know what a syllabus is about if you're an academic. You
  have taken classes and have read other people's syllabi. You might
  have written your own. This is a comfortable format.\\
\item
  When making your own classes you probably look at other people's
  syllabi and ``borrow'' from them (with attribution of course). This is
  not only acceptable, it's normal in most fields. Same with this
  document. Take what you want, leave the rest. Some of this will be
  useful, some will not.
\item
  Leaving academia and learning the skills to make this leap is no
  different than any other skill you've learned in a class. It's just
  work. A syllabus reminds you that this is doable. Plus I get to give
  you homework and who doesn't like homework?
\item
  A syllabus changes from semester to semester. This document will do
  the same, staying online and shifting as I get more resources,
  suggestions for units and ideas for how best to format this thing. I
  strongly encourage you to contribute to this document as we go.
\end{enumerate}

\section{Prerequisites for the
course}\label{prerequisites-for-the-course}

This syllabus will make the most sense for people who have spent time in
a graduate program that is primarily designed to prepare you for a
career in academia. This course also presumes that you are open to a
life that is somehow different from becoming a tenured professor at a
university.

A few things that are not required: You don't have to have left
academia, you don't have to make a decision about if you'll leave. You
don't even have to know what that would look like if you did.

\section{A few Terms}\label{a-few-terms}

\begin{itemize}
\tightlist
\item
  Post Academic/Post Ac
\item
  Academic Adjacent/Ac Adjacent
\item
  Adjuncts
\item
  Postdoc/Postdoctoral Fellowship
\item
  VAP/Visiting Assistant Professor Position
\item
  TT/Tenure Track Position
\end{itemize}

\section{Contributing}\label{contributing}

I take responsibility for this document and the opinions are mine unless
they are in quotations and therefore contributed by someone else. That
said, I strongly encourage you (yes you!) to contribute resources,
materials, feedback about books/articles/resources, ideas for
assignments, or any other thoughts you have. I welcome emails, responses
to this survey and pull requests on GitHub. I reserve the right to edit
for clarity, space and to spelling/grammar. I also may not be able to
fit everything into the document, but I will do my best.

Please be patient if you do submit something. I am one person, and this
is a labor of love. I also don't have all the answers for your field or
your experiences. So if something seems wrong to you, please let me
know. We all need each other here. There's a lot we can accomplish if we
join together our (massively overeducated) brains.

If I do include something that you have contributed, you have the option
of having your name listed in the contributors section with my
gratitude. Making your name public is not required for contributing.

\section{About}\label{about}

This syllabus is designed by me Beth Duckles (Hi!), but it would not
exist without the incredible group of women who have joined the Athenas
Slack Channel for Post Ac women, the people who responded to a survey I
put out for resources on Post Ac life and the many contributors who have
offered their thoughts and resources.

\chapter{Quit Lit}\label{quit-lit}

Quit lit is rarely defined but most academics have seen or heard of its
prominence in recent years. These documents are a particular type of
discussion of the challenges in higher education. They often take a
first person perspective and reflect both on the individual's story and
how this connects to the larger questions within the academy and higher
education. By starting with an incomplete typology of perspectives on
why people leave the academy, we get a sense of the variety of stories
that are told and can reflect on how these themes may emerge in our own
stories.

The point is to begin to own one's story and to take seriously how our
own individual positions and experiences are connected to the larger
world. Being willing to examine our own story in light of the larger
structure, industry and mechanisms that we have been a part of can ease
the individual burden and begin the healing process.

\section{Readings}\label{readings}

\begin{itemize}
\tightlist
\item
  \href{https://www.theatlantic.com/entertainment/archive/2015/09/dont-quit-your-day-job/404671/}{Garber,
  Megan. ``The Rise of Quit Lit'' - Atlantic}
\item
  \href{http://www.slate.com/articles/life/culturebox/2013/04/there_are_no_academic_jobs_and_getting_a_ph_d_will_make_you_into_a_horrible.html}{Schuman,
  Rebecca ``Thesis Hatement'' - Slate}
\item
  \href{http://erinbartram.com/uncategorized/the-sublimated-grief-of-the-left-behind/}{Bartram,
  Erin - ``The Sublimated Grief of the Left Behind''}
\item
  \href{http://www.alicolleenneff.com/blog/2017/11/8/on-academic-precarity}{Neff,
  Ali Colleen - ``On Academic Precarity''}
\item
  \href{\%20https://www.vox.com/2015/9/8/9261531/professor-quitting-job}{Lee,
  Oliver - ``I have one of the best jobs in Academia. Here's why I'm
  walking away'' - Vox}
\item
  \href{\%20https://conditionallyaccepted.com/2015/06/16/quit/}{Conditionally
  Accepted, ``Dear Department, I Quit.''}
\item
  \href{\%20https://www.allisonharbin.com/post-phd/why-i-left-academia-part-1}{Harbin,
  Alison - ``Why I Left Academia Part I \& II''}
\item
  \href{http://erinbartram.com/uncategorized/the-sublimated-grief-of-the-left-behind/}{Erin
  Bartram - ``The Sublimated Grief of the Left Behind''}
\item
  \href{https://chroniclevitae.com/news/216-why-so-many-academics-quit-and-tell}{Dunn,
  Sydni - ``Why So Many Academics Quit and Tell''}
\item
  For more, see the list of pieces in a Google
  Doc:\href{\%20https://docs.google.com/spreadsheets/d/1OODoiZKeAtiGiI3IAONCspryCHWo5Yw9xkQzkRntuMU/edit\#gid=0}{''Quit
  Lit: The Vitae List''}
\end{itemize}

\section{Homework:}\label{homework}

\begin{enumerate}
\def\labelenumi{\arabic{enumi}.}
\tightlist
\item
  After reading the above quit lit, find more stories that fit with your
  experiences. Consider the following themes and note which ones fit
  best with your story.
\end{enumerate}

\begin{itemize}
\tightlist
\item
  academic precarity and economic instability
\item
  the contingent labor market and low-paid adjunct positions
\item
  not enough tenure track jobs and the competition for tenure track
  positions
\item
  academia is not for me
\item
  academia is for me but I didn't get a job
\item
  academia is for me but I hate teaching
\item
  unfair teaching burdens
\item
  student apathy
\item
  student entitlement
\item
  unprepared students
\item
  student as customer
\item
  the rise of online education
\item
  increased student tuition
\item
  student loan bloat
\item
  the broken hiring process
\item
  the broken tenure process
\item
  the broken academic publishing process
\item
  decreased university funding
\item
  increased higher education administration
\item
  low faculty salaries
\item
  declining funding for research
\item
  the declining liberal arts
\item
  anti-intellectualism
\item
  classism
\item
  geographic isolation
\item
  loneliness
\item
  grief
\item
  the ``two body'' problem
\item
  having a child
\item
  having a child with special needs
\item
  having small children
\item
  incompatible careers
\item
  family illness
\item
  divorce
\item
  everyday institutional racism
\item
  microaggressions
\item
  mental illness
\item
  workaholism
\item
  everyday sexism
\item
  sexual harrassment
\item
  sexual assault
\item
  assault
\item
  verbal abuse
\item
  gaslighting
\item
  unequal emotional labor workload
\item
  illegal behavior
\item
  institutional politics and infighting
\item
  sabotage
\item
  the trap of post doctoral positions
\item
  the trap of visiting assistant positions
\item
  the trap of adjuncting
\item
  dropping out of a Ph.D program
\item
  the desire to do work that matters
\item
  the desire to do manual work
\item
  the desire to have a life.
\end{itemize}

\begin{enumerate}
\def\labelenumi{\arabic{enumi}.}
\setcounter{enumi}{1}
\tightlist
\item
  Write your own quit lit piece. If you have not already left, imagine
  you have or will soon leave. If you have left, make your writing
  cathartic.
\end{enumerate}

Optional Extra credit: Let someone (or a lot of someones) read your quit
lit piece.

\chapter{You Are Not Alone}\label{you-are-not-alone}

This unit reminds you that regardless of your experience, there are
others who have had similar experiences and who have left the academy.
There is even some evidence that you will (gasp) be happier if you
leave.

There are a myriad of people out there who are doing work that will help
you even if you have no money, there are books you can check out from
the library, concrete steps you can take and people who will talk to
you.

We will start with general resources and then follow up with specific
resources for certain groups.

\section{Readings}\label{readings-1}

\begin{itemize}
\tightlist
\item
  What I wish I had Known - Beth M. Duckles
\item
  \href{http://theprofessorisin.com/}{The Professor is In}
\item
  \href{https://www.imaginephd.com/}{Imagine Ph.D.}
\item
  \href{https://community.beyondprof.com/}{Beyond the Professoriate}
\item
  \href{https://www.timeshighereducation.com/blog/what-happens-when-academics-quit-good-things-it-turns-out}{Perel,
  Greta - ``What Happens When Academics Quit? Good Things it Turns
  Out.''}
\end{itemize}

\section{Resources for Women}\label{resources-for-women}

Women often face very specific challenges with leaving the academy.
There are a number of resources that may help you work with and heal
from these challenges as well as assisting you in getting your story
back into your own hands.

\begin{itemize}
\tightlist
\item
  \href{https://www.taramohr.com/book/}{Tara Mohr, Playing Big}
  \textgreater{} ``Gave me lots of solid information about women and
  ambition, and eventually connected me to smart women also trying to
  figure shit out.
\end{itemize}

\section{Resources for People of
Color}\label{resources-for-people-of-color}

\section{Doing your Human Homework}\label{doing-your-human-homework}

Leaving the academy is a huge life shift and as the child of two
psychologists, I'm keen to encourage people to work through the things
that have brought them emotional pain. Basically, I want you to do your
human homework.

Human Homework is the act of facing and working with your fear, shame
and emotions so that you can heal. It's not something you ever finish
doing but it is something that you need when you're going through
difficult times.

Doing your human homework regardless of who you are will make you a more
effective and whole human being. I strongly believe that you are not
broken and the world has need of your intelligence and skills.

I do not have opinions about what specific method would work best for
you to do this work. I just encourage you to do it.

Below is a list of resources that might help. Some might intrigue you,
while others may turn you off. Just find the resources or method that do
speak to you and work with those.

\begin{itemize}
\tightlist
\item
  Transitions Book - William Bridges
\item
  Nonviolent Communication
\item
  Sedona Method
\item
  \href{http://www.couragerenewal.org/}{Parker Palmer and the Center for
  Courage and Renewal} A former academic and writer, Parker writes about
  his experiences and share insights from the Quaker tradition and his
  time with the retreat center Pendle Hill. His books reflect on how to
  ``let your life speak'' and to look for wholeness in your life. He
  works with the Center for Courage and Renewal to create retreats and
  other seminars.\\
\item
  {[}Brene Brown{]} Brene Brown is a professor of social work at UT
  Austin who did a TEDx talk that went viral about vulnerability. She
  has written several books based on her research. Among them are Rising
  Strong
\item
  Pema Chodron
\item
  Landmark Forum
\item
  Last Mask Shamanic Center
\item
  More to Life
\item
  The Road Less Traveled - Peck
\item
  \href{http://www.leadershipembodiment.com/}{Leadership Embodiment -
  Wendy Palmer}
\item
  The Big Leap (???)
\item
  \href{https://marthabeck.com/}{Martha Beck} Martha is a former
  academic who has written memoirs, self help books and has created a
  tribe of people who are focused on finding direction. Among her books:
  Steering By Starlight, Finding Your Way in a Wild New World.
\item
  \href{https://en.wikipedia.org/wiki/Man\%27s_Search_for_Meaning}{Man's
  Search for Meaning - Viktor Frankl} A book written by a psychologist,
  chronicling his experiences in Auchwitz and finding meaning in life.
\item
  \href{http://www.jonathanfields.com/}{``Uncertainty: Turning Fear and
  Doubt into Fuel For Brilliance - Jonathan Fields}
\item
  \href{https://www.sharonsalzberg.com/}{``The Power of Meditation: A 28
  Day Program - Sharon Saltzberg}
\item
  \href{https://www.harrietlerner.com/}{The Dance of Anger - Harriet
  Lenrner}
\end{itemize}

\chapter{Networking}\label{networking}

This class will demystify and simplify the task of networking by
encouraging you to note your strengths (yes introverts have networking
strengths) and to give you a chance to play around with and consider how
best to network into new ideas.

The truth is, that getting a new job, starting a new career, or even
being a public academic requires networking. I am not suggesting that
you change who you are, or do anything that turns you into a smarmy
salesperson. You gotta be you.

\section{Resources}\label{resources}

\begin{itemize}
\tightlist
\item
  \href{https://code.likeagirl.io/networking-strategies-for-beginners-986fe0b3efdf}{``Bajuniemi,
  Abby - ``Networking Strategies for Beginners''}
\item
  Linkedin
\item
  Meetup.com
\item
  Twitter
\item
  Facebook Groups
\end{itemize}

\section{Homework}\label{homework-1}

\begin{enumerate}
\def\labelenumi{\arabic{enumi})}
\tightlist
\item
  Create social media profiles for Linked In, Twitter and Meetup. If you
  dislike or are nervous about being on one or more of these platforms
  consider creating very specific boundaries around how you choose to
  use them. For instance, you might decide that you will not friend
  people on Linked in that were in your classes. Or you might only tweet
  things that you think people who are in your field would be interested
  in. Or you might decide
\end{enumerate}

2a) Connect with five new people on LinkedIn that you already know.
Write them a short personal note saying something along the lines of:
``It's great to see you on here, hope you are doing well. - YourName''
or ``I was thinking about you the other day and it's good to look you up
on here and see you're doing well.'' - YourName or ``I saw this
article/anecdote/event/meme and thought of you. Hope you're doing
well.'' - YourName

2b) Follow someone you think is interesting on Twitter. Read their
tweets.

2c) Look for some Meetups you might consider going to. Put one on the
calendar.

2d) Join a new Facebook Group and respond to or ``like'' someone's post.

\chapter{Getting a Job}\label{getting-a-job}

There are a thousand and one books out there on how to find a job and a
ton of people who can help. That said, it can be somewhat overwhelming
to approach those books when you're a post academic because you have
some very specific challenges.

You might wonder: - How do I turn my CV into a resume? - How do I talk
about my research/papers/work to non academics? - What skills exactly do
I have? - How do I market my skills? (and btw marketing is a dirty word)
- Who would hire me? - What if I want to do something completely
different that I'm entirely unprepared for?

First, recognize that you have skills. Even if you have the most
esoteric dissertation in the world, you have skills and often can teach
them or use them in service to all kinds of projects. Consider what you
did love about academia. Was it reading? Writing? Teaching? Was it the
content you were working on? Was it the methods you were using? Did you
like the meetings or prefer to hide out in the lab/library/archives?
Consider your answers to these questions as you look for your job and
try to find places where those skills are valued.

Second, the path of finding your post ac job can be trial and error. It
often does not follow a clear path and there isn't anyone else who can
know what the next step is for you to take. It can be bewildering,
frustrating and incredibly scary for academics who have had the path
laid out for them (take these classes, then do these exams, then write
this paper etc). This kind of figure-it-out-as-you-go mentality is
actually normal for non academics. If you need to be reminded of this,
go talk to a friend who never went to grad school and ask them how they
got their current job. You might be surprised.

Third, you gotta pay your bills. Don't discard jobs that are completely
different than your academic work, if they fit for you and make you
happy. There is no shame in doing work that is non academic to make ends
meet. Do side gigs if you need to, start up an etsy shop, work with high
school kids on their SAT prep, teach art classes, do freelance writing
or editing. Consider arranging flowers, being a cheesemonger, becoming a
sommelier, a bike mechanic, doing community organizing, running online
seminars, Let me reiterate, there is no shame in any of these jobs. Or
rather, you can feel ashamed if you really want to but you might
consider that most people are not judging you.

\section{Resources}\label{resources-1}

\subsection{Books Focused on
Academics}\label{books-focused-on-academics}

\begin{itemize}
\tightlist
\item
  \href{https://www.amazon.com/What-Are-You-Going-That/dp/0374526214}{So
  What Are You Going To Do With That? - Susan Basalla}
\item
  \href{http://theprofessorisin.com/buy-the-book/}{The Professor is In -
  Karen Kelsky}
\end{itemize}

\subsection{Websites}\label{websites}

\begin{itemize}
\tightlist
\item
  \href{http://jobsontoast.com/}{Jobs on Toast}
\item
  \href{https://versatilephd.com/}{Versatile Ph.D.}
\end{itemize}

\subsection{Books on Offbeat Jobs}\label{books-on-offbeat-jobs}

\begin{itemize}
\tightlist
\item
  \href{https://bornforthisbook.com/}{Born for This: How to Find the
  Work You Were Meant to Do - Chris Guillibeau}
\item
  \href{https://www.harpercollins.com/9780062566652/how-to-be-everything/}{How
  to Be Everything: A Guide for Those Who (Still) Don't Know What They
  Want to Be When they Grow Up}
\item
  \href{https://www.tessvigeland.com/}{Leap: Leaving a Job with No Plan
  B to Find the Career and Life You Really Want - Tess Vigeland}
\item
  \href{https://herminiaibarra.com/books/}{Working Identity:
  Unconventional Strategies for Reinventing your Career - Herminia
  Ibarra}
\item
  \href{https://chrisguillebeau.com}{The Art of Nonconformity - Chris
  Guillibeau}
\item
  \href{https://www.harpercollins.com/9780062472724/weird-in-a-world-thats-not/}{Weird
  in a World That's Not: A Career Guide for Misfits, F*ckups, and
  Failures by Jennifer Romolini}
\end{itemize}

\subsection{Articles}\label{articles}

\begin{itemize}
\tightlist
\item
  \href{https://medium.com/design-leadership/benefits-of-a-non-linear-design-career-path-d0add49a90c8}{Benefits
  of a Non Linear Career Path}
\item
  \href{https://www.insidehighered.com/news/2017/12/18/study-humanities-and-social-science-phds-working-outside-academe-are-happier-their?utm_source=Inside+Higher+Ed}{Post
  Acs are Happier than Academics}
\end{itemize}

\subsection{Entrepreneurship}\label{entrepreneurship}

\begin{itemize}
\tightlist
\item
  \href{https://www.amazon.com/Disrupt-Yourself-Putting-Disruptive-Innovation-ebook/dp/B00VQOA26Q}{Disrupt
  Yourself: Putting the Power of Disruptive Innvoation to Work - Whitney
  Johnson}
\item
  \href{https://www.amazon.com/dp/B0067TGSOK/ref=dp-kindle-redirect?_encoding=UTF8\&btkr=1}{\$100
  Startup - Chris Guillibeau}
\item
  \href{https://sidehustleschool.com/}{Side Hustle: From Idea to Income
  in 27 Days - Chris Guillibeau Book and Side Hustle School}
\end{itemize}

\subsection{Marketing}\label{marketing}

\begin{itemize}
\tightlist
\item
  \href{https://www.amazon.com/Book-Yourself-Solid-Reliable-Marketing/dp/0470643471}{Book
  Yourself Solid - By Michael Port}
\end{itemize}

\section{Homework}\label{homework-2}

\begin{enumerate}
\def\labelenumi{\arabic{enumi}.}
\tightlist
\item
  Write down ten jobs you would like to do. Don't think about it, just
  write them down. Doesn't matter if they're silly or strange. Feel free
  to put circus performer, professional basketweaver or person who does
  nature walks for a living. Do this every day for a week.
\item
  Look up the concept of
  \href{https://en.wikipedia.org/wiki/Ikigai}{Ikigai} or your ``Reason
  for Being''. Print out a copy of the diagram (a google search here
  will help) and fill in your skills/projects so that you can see where
  your skills fit. Ask questions of the things that you fill in. For
  instance, if one task fits into the category of What you love and what
  the world needs, how might you find a way to also get good at the task
  and/or to get paid for it?\\
\item
  Ask someone you trust to tell you what they think you're good at.
  Listen without interrupting. Write down what they say.
\end{enumerate}

\chapter{Creativity}\label{creativity}

It may seem strange for me to have a week on creativity but in my own
recovery from academia I've found it to be an incredibly potent tool for
dealing with what I feel and think as well as a way to feel more content
and happy in life.

I believe we are all creative. I don't think there are any among us who
are not creative. For many academics, the creative urges have been
quashed by the desire to please those who we have looked up to in our
school years. We may also believe ourselves not to be creative, or to
have taken to hear the kind of language that tells us that a ``true''
scientist/historian/professor isn't a creative person.

My number one recommendation is Julia Cameron's ``The Artist's Way''. In
that book she describes much of what has caused many of us academics a
lot of pain. She does so with compassion and grace and she slowly,
steadily and lovingly encourages the nascent artists inside of us out of
their shell.

\begin{itemize}
\tightlist
\item
  Julia Cameron ``The Artist's Way''
\item
  SARK's book
\item
  Ignore Everybody - MacLeod
\item
  Art and Fear
\end{itemize}

\section{Homework}\label{homework-3}

\begin{itemize}
\tightlist
\item
  Do one thing that seems fun and creative. A few ideas: Play with
  legos, cook a meal, doodle on a piece of scrap paper until the page is
  filled up, write bad poetry about something you can see in the room
  you're in now, take a train, take a pottery class, bus or car trip and
  spend the entire trip looking out of the window, hum a song, play an
  instrument, drum on the tabletop, paint, garden, sew, knit, crochet,
  find a youtube video that teaches you a dance step, make snow/sand
  angels, make splashes in the water and watch the shape they make, eat
  your favorite childhood snack, walk somewhere interesting, skip
  stones, build a little teepee out of sticks in the woods, make a fire,
  do karaoke, read a poem or a speech out loud, watch a movie, make
  paper airplanes, play with arduino, go see a play, do a zumba class,
  learn a few words in another language, watch a dance performance and
  anything else your heart desires.
\end{itemize}

\chapter{Entrepreneurship for Post
Academics}\label{entrepreneurship-for-post-academics}

Freelancers Union
\href{https://www.freelancersunion.org/}{~{[}{]}https://www.freelancersunion.org/}
- Being Boss Podcast - Your local small business development council
(SBDC) - SCORE office \url{https://www.score.org/} - \$100 Startup
Marketing - Seth Godin - Books/Marketing Seminar/Akimbo Podcast -
Marketing Mentor Podcast - We Write You - How to Write Like a Person\\
- Leonie Dawson's Workbook - Biz
\url{https://shiningacademy.com/2018-life-and-business-goals-workbooks-and-diary-planners-by-leonie-dawson/}
Tools - \url{https://app.and.co/}

\chapter{Industry Specific Resources}\label{industry-specific-resources}

\section{Writing/Publishing}\label{writingpublishing}

\begin{itemize}
\tightlist
\item
  \href{www.loft.org}{Loft Writing Classes}
\item
  \href{www.pw.org}{WHAT IS THIS?}
\item
  \href{https://www.authorspublish.com/}{Authors Publish}
\item
  \href{http://www.theinternationalfreelancer.com/}{The International
  Freelancer}
\end{itemize}

\section{Linguistics}\label{linguistics}

\begin{itemize}
\tightlist
\item
  {[}Career Linguist(www.careerlinguist.com)
\end{itemize}

\section{\#\# Data Science}\label{data-science}

\section{Public Policy}\label{public-policy}

\begin{itemize}
\tightlist
\item
  AAAS Fellowship
\end{itemize}

\section{User Experience/User Design}\label{user-experienceuser-design}

\begin{itemize}
\tightlist
\item
  18F
\item
  IDEO
\item
  Stanford D School
\end{itemize}

\section{Social Scientists}\label{social-scientists}

\begin{itemize}
\tightlist
\item
  \url{https://www.epicpeople.org/}
\end{itemize}

\section{Other Resources}\label{other-resources}

\begin{itemize}
\tightlist
\item
  herc.com
\item
  The Artful Adjunct
\item
  The Art of Nonconformity
\end{itemize}

\chapter{Money}\label{money}

Money is an incredibly difficult subject for post acs.

\subsection{Personal Finance
Resources}\label{personal-finance-resources}

\begin{itemize}
\tightlist
\item
  \href{https://www.harpercollins.com/9780061856662/overcoming-underearningtm/}{Overcoming
  Underearning - Barbara Stanny}
\item
  \href{http://thefinancialdiet.com/author/chelsea-fagan/}{Financial
  Diet - Chelsea Fagan}
\item
  \href{http://shop.oreilly.com/product/9780596809416.do}{Your Money The
  Missing Manual - JD Roth}\\
\item
  \href{https://www.penguinrandomhouse.com/books/20308/money-drunkmoney-sober-by-mark-bryan-and-julia-cameron/9780345432650/}{Money
  Drunk/Money Sober - 90 Days To Financial Freedom by Mark Bryan and
  Julia Cameron}
\end{itemize}

\subsection{Websites}\label{websites-1}

\begin{itemize}
\tightlist
\item
  \href{http://pfforphds.com/}{Personal Finance for PhDs}
\item
  \href{https://www.mint.com/}{Mint}
\item
  \href{https://www.youneedabudget.com/}{You Need a Budget (YNAB)}
\item
  \href{http://www.simplelivingforum.net/}{The Simple Living Forum}
\item
  \href{https://www.getrichslowly.org/}{Get Rich Slowly}
\end{itemize}

\section{Being Broke}\label{being-broke}

\begin{itemize}
\tightlist
\item
  \href{https://www.amazon.com/Rising-Strategies-Broke-At-Risk-Those/dp/151874043X/}{Rising:
  Strategies for the Broke, the At-Risk, and Those Who Love Them by Joon
  Madriga}
\item
  \href{https://www.amazon.com/dp/B00I9DOSDU/ref=dp-kindle-redirect?_encoding=UTF8\&btkr=1}{Poorcraft:
  The Funnybook Fundamentals of Living Well on Less}
\item
  \href{https://www.amazon.com/dp/B00JPR5JA0/ref=dp-kindle-redirect?_encoding=UTF8\&btkr=1}{Hand
  to Mouth: Living in Bootstrap America by Linda Tirado}
\end{itemize}

\section{Unique Money Perspectives}\label{unique-money-perspectives}

\begin{itemize}
\tightlist
\item
  \href{https://yourmoneyoryourlife.com/}{Your Money or Your Life -
  Vicki Robin and Joe Dominguez} The perspectives in this book really
  can change the way you think about money. Through several exercises
  they get at the question of how much your time and life force is
  worth. This book can be quite eye opening and is a very unique way to
  consider how money fits into your life. There is also a vast internet
  community of folks who have worked through this book and consider
  their lives based on these ideas.\\
\item
  \href{http://sacred-economics.com/}{Sacred Economics - Charles
  Eisenstein} Eisenstein asks for no less than a reconsideration of what
  we think of as the use of money. He is interested in questioning how
  we look at the economic system from gift economies to capitalism. He's
  interested in questioning the capitalist system through a deeper
  inquiry into how we might exist in a gift economy.
\end{itemize}

\section{Money for Entrepreneurs}\label{money-for-entrepreneurs}

\begin{itemize}
\tightlist
\item
  \href{http://www.singlefounder.com/tips-on-negotiating-a-great-consulting-rate/}{Consulting
  and Negotiation}
\end{itemize}

\bibliography{book.bib,packages.bib}


\end{document}
